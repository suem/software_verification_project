\section{Boogie}\label{boogie_discussion}

In this part we discuss the second part of the project, implementing and verifying quicksort and bucketsort in Boogie.\\
First we describe how we specified our algorithms i.e. how we described the complete behaviour of the two sorting algorithms. We then quickly discuss how the two algorithms were implemented followed by a discussion about the verification task. There we describe the various problems we faced and how we overcame them.

\subsection{Algorithm Specification}

Both quicksort and bucketsort share the same specification aside from some minor differences due to implementation details. Therefore we just state the specification of a general sorting algorithm in Boogie which 
applies to both quicksort and bucketsort.\\

The signature of our sort procedure looks as follows:

\begin{verbatim}
  procedure xSort(lo: int, hi: int) returns (perm: [int]int)  
    modifies a;
   ...
   {}
 \end{verbatim}

The sort procedure sorts a global variable \texttt{a:[int]int} from indices \texttt{lo} to \texttt{hi} (both, \texttt{lo} and \texttt{hi}, inclusive).
This global variable is a map which models a one-dimensional array of integers. 
We decided to specify that the sorting algorithm works on an arbitrary sub-sequence of the array because it results in a more general algorithm and because it makes implementing quicksort's recursivitiy easier.
The return value of \texttt{xSort} is a map from \texttt{int} to \texttt{int} which represents a permutation on the
array indices that describes how the elements of the array were swapped by \texttt{xSort}.
\\

\texttt{xSort} has only one precondition, namely that the input range must not be negative:
\begin{verbatim}
  requires lo <= hi;
\end{verbatim}

To specify the entire behaviour of the sort procedure, we state the following postconditions.
\begin{itemize}
\item After the procedure call, the array must be sorted:
  \begin{verbatim}
  ensures (forall k, l: int :: 
    lo <= k && k <= l && l <= hi ==> a[k] <= a[l]
  );
\end{verbatim}

\item The return value must be a valid permutation of the array indices (from \texttt{lo} to \texttt{hi}).
  \begin{verbatim}
  ensures (
    forall i: int :: 
      lo <= i && i <= hi 
      ==> lo <= perm[i] && perm[i] <= hi
  );
  ensures (
    forall k, l: int :: 
      lo <= k && k < l && l <= hi 
      ==> perm[k] != perm[l]
  );
\end{verbatim}

\item The final array is a permutation of the input array. This permutation is given by the return value \texttt{perm}.
  \begin{verbatim}
  ensures (
    forall i: int :: 
      lo <= i && i <= hi 
      ==> a_qs[i] == old(a_qs)[perm[i]]
  );
\end{verbatim}
 
\end{itemize}

This specification is based on the bubblesort example that is 
given on the Boogie web page \footnote{\url{http://rise4fun.com/Boogie/Bubble}}. We decided to 
copy the approach of using a concrete permutation to state that the final array is a permutation of
the input array because we felt that it would be easier to actually construct such a permutation rather
than having to prove the existence of one.

\subsection{Quicksort Implementation}

We implemented the textbook definition of quicksort as described in "Introduction to Algorithms"\cite{Cormen}. 
Quicksort is implemented in the \texttt{qs} procedure. The pivot element is set to be the rightmost element of the array.\\
First the algorithm partitions the array into a left and right part such that the left part is smaller than or equal to the pivot and the right part is greater than the pivot.\\
This partition step is implemented in a separate procedure \texttt{qsPartition}.\\
After partitioning, \texttt{qs} is called recursively on the left and the right part of the array. Here
the fact that we specified the sorting procedures to work on arbitrary parts of the array comes in handy.\\

In addition to the actual sorting, \texttt{qs} and \texttt{qsPartition} both construct a permutation that
keeps track of how the array's elements are swapped. A big part of the implementation consists of code that
combines the permutations returned by \texttt{qsPartition} and the two recursive calls to \texttt{qs} into a single permutation
capturing all swap operations.

\subsection{Bucketsort Implementation}

Bucketsort is implemented in the \texttt{bucketSort} procedure.  
The procedure divides the array into three buckets with elements in the range of $[-3N, -N)$, $[-N,N)$ and
$[N,3N]$ respectively. 
We decided to use these ranges because we know that the procedure will only be called with elements from $[-3N,3N]$ and given no further information about them, we can but optimistically assume they are uniformly distributed over that range.\\
These three buckets are then each sorted individually using quicksort.\\
The now sorted buckets are then copied back into the original array yielding a sorted version
of the original array.
\\\\
Just as in quicksort, the bucketsort implementation also has to construct a permutation that reflects
the swap operations of the algorithm. This was more challenging than in quicksort, because the size of the
three buckets is not known in advance.

Note that being able to call quicksort with three buckets was not a trivial task because of some limitations
of the Boogie language. This is addressed in the next section (\ref{sec:mod_sort}).

\subsection{Implementing modular sorting algorithm}
\label{sec:mod_sort}

In Boogie, as opposed to Eiffel/AutoProof, it is not possible for procedures to modify input
arguments. This is why our sorting algorithm is specified to work on a global variable. 
In the case of bucketsort however it is necessary to sort three different arrays with the
same algorithm (quicksort). This however is not possible because the \texttt{qs} procedure
can only sort one global variable (\texttt{a}). To solve this issue, we first modified our quicksort implementation
to work on a different global variable \texttt{a\_qs} and then created the following procedure:

\begin{verbatim}
procedure quickSort(arr : [int]int, lo : int, hi : int) 
  returns (arr_sorted : [int]int, perm: [int]int) 
  ...
{
  a_qs := arr;
  call perm := qs(lo,hi);
  arr_sorted := a_qs;
}
\end{verbatim}

The new \texttt{quickSort} has the same specification as \texttt{qs} but instead of sorting a global array,
it takes an array as argument and returns a sorted copy. This is achieved by using \texttt{a\_qs} as temporary 
variable to store the input array. Now with this construction it is possible to sort arbitrary arrays using
our existing quicksort implementation.\\

An alternative approach would have been to directly modify the \texttt{qs} implementation to take an array as argument
and return a sorted copy of it. We decided against this solution because it would have required additional code that
takes care of copying the values from input to output parameters. This would then add additional complexity 
to the verification task of the project. Our solution has the advantage that it allowed us to reuse our existing quicksort
implementation with only minor modifications.


\subsection{Verification}

We managed to verify all our procedures with Boogie. However this was not an easy task. While working 
with boogie we encountered several difficulties which will be discussed in this section.\\

A general problem that we faced in various places was the following:\\
Assume that after a while loop, we have proven a property to hold for a section of an array. A second while loop that modifies a different section of the same array
would then erase this information about the first section.\\
We solved this problem by always adding all properties that
had been previously shown about an array into the loop invariant of each subsequent while loop that would modify 
said array again. Unfortunately this lead to rather verbose loop invariants which would be repeated in several loops.\\

A variant of this problem also occurred while verifying quicksort.\\
Because each recursive call to \texttt{qs} modifies the same global variable \texttt{a\_qs}, all previous information about \texttt{a\_qs} 
are lost after the procedure call. This was a problem because we would loose the information that the left
part of the array is sorted after calling quicksort on the right part. 
We solved the issue by adding the following postcondition to \texttt{qs} and subsequently also \texttt{qsPartition}.

\begin{verbatim}
  ensures (forall i: int :: i < lo || i > hi
  		==> a_qs[i] == old(a_qs)[i]);
\end{verbatim}

The postcondition above states that only the elements in the range given by $[lo, hi]$ are modified. Proving this
property was straightforward and it resolved the problem that was described above. Using this information, 
Boogie was able to infer that all previous properties about different parts of the array still hold.\\

Another big problem was the fact that the Boogie verifier would not always terminate after a 
sensible amount of time. Furthermore, having the verifier terminate could be influenced by
adding certain assertions to the code. In fact, our final verification of quicksort requires
two assertions to be present for the verifier to terminate. We highlighted these assertions in our
source code using comments.\\

The biggest problem we faced was during the verification of bucketsort. There we had to show
that if a sorted array (a bucket) is copied into another array (the original array to sort), 
then this part of the array is also sorted. Proving this boiled down to the following lemma:

\begin{verbatim}
procedure lemma1(a : [int]int, b:[int]int, off : int, n : int) 
    requires off >= 0;
    requires n >= 0;
    requires (
      forall k: int :: 0 <= k && k < n 
        ==> a[k+off] == b[k]
    );
    requires (
      forall k, l: int :: 0 <= k && k <= l && l < n 
        ==> b[k] <= b[l]
    );
    ensures (
      forall k, l: int :: 0 <= k && k <= l && l < n 
        ==> a[k+off] <= a[l+off]
    );
{
    assume (
      forall k, l: int :: 0 <= k && k <= l && l < n 
        ==> a[k+off] <= a[l+off]
      );
}
\end{verbatim}

Here \texttt{a} is the original array and \texttt{b} is the bucket array. The lemma states that 
if the bucket \texttt{b} is sorted and if the array \texttt{a} is equal to \texttt{b} in
the index range from \texttt{off} to \texttt{off+n} then \texttt{a} is also sorted in this range.
However boogie is not able to prove this lemma not even by assuming the postcondition in the
procedure body.\\

In the end we figured out that the problem came from the way the array \texttt{a} was indexed.
The lemma above uses a constant offset \texttt{off} to address a certain part in \texttt{a}. 
By reformulating this problem into a form that doesn't use this offset, Boogie is able
to prove the lemma:

\begin{verbatim}
procedure lemma2(a:[int]int, b:[int]int, lo:int, hi:int, n:int) 
    requires hi-lo+1==n;
    requires n >= 0;
    requires (forall k: int :: lo <= k && k <= hi 
    	==> a[k] == b[k-lo]);
    requires (forall k, l: int :: 0 <= k && k <= l && l < n 
    	==> b[k] <= b[l]);
    ensures (forall k, l: int :: lo <= k && k <= l && l <= hi 
    	==> a[k] <= a[l]);
{ }
\end{verbatim}

Figuring out this trick took us a long time.\\

Another problem we faced is the fact that Boogie is not always correct. We were able to have Boogie prove 
a false statement. Boogie would verify the following procedure:

\begin{verbatim}
const N: int;
axiom 0 <= N;
procedure gaga()
  ensures (forall k,l:int :: 0<=k && k<=l && l < N ==> false); 
{
  var x:int;
  var i :int;
  x := -N;
  i := 0;
  while(i < N) { 
    i := i + 1;
  }
}
\end{verbatim}

This turned out to be a bug which has since been fixed\footnote{\url{https://github.com/boogie-org/boogie/issues/25}}.\\

The various problems described above made the verification task rather frustrating compared to the
first part of the project. In section \ref{sec:conclusion} we'll discuss the differences between working with 
AutoProof and Boogie.


