\section{Boogie}\label{boogie_discussion}
TODO
\subsection{Quicksort Implementation}

We implemented quicksort as specfified in "Introduction to Algorithms (TODO cite)". 
The design of our boogie implementation was mostly influenced by the bubblesort example 
from microsoft research (TODO cite). A global map variable \texttt{a :[int]int} is used to represent the
array to be sorted. The qucksort implementation modifies this variable.

Our implementation of quicksort is divided into the following two procedures:
\begin{verbatim}
procedure qsPartition(lo : int, hi : int) 
  returns (pivot_index: int, perm: [int]int) {
  ...
}
\end{verbatim}

\begin{verbatim}
procedure qs(lo : int, hi : int) returns (perm: [int]int) {
  ...
}
\end{verbatim}

Both procedures take arguments \texttt{lo} and \texttt{hi} to specify which part of the array is processed.
\texttt{qsPartition} divides the array into two parts, a left part that is smaller than or equal to the pivot
element and a right part that is greater than the pivot element. The pivot element is defined to be the right most element of the array. 
The return value of \texttt{qsPartition} is the final index of the pivot element and another map \texttt{perm :[int]int} which is a permutation
on the range $lo$ to $hi$. This permutation expresses how the original array was permutetd by the algorithm.
\\\\
The second procedure \texttt{qs} first calls \texttt{qsPartition} on the entire array, 
then it recursively calls itselfe on the left and the right part of the array. This procedure also returns 
a permutation to indicate how the elements of the array were permuted.
\\\\
Implementing the actual sorting algorithm was actually rather easy. We could basically copy the textbook definition to implement
\texttt{qsPartition} and \texttt{qs}. The hard part was to construct a permutation that keeps track of how the array elements are mutated.
In \texttt{qsPartition} this part is easy because the \texttt{perm} map just keeps track of how array elements are swapped. The difficult part was to
combine the permutation that is return by \texttt{qsPartition} with the two permutations that are returned by the two recursive calls to \texttt{qs}.
First the two permutations returned by the recursive calls have to be combined to a new permutation on the entire range from $lo$ to $hi$. Then this permutation
is again combined with the permutation returned by \texttt{qsPartition}. This bookkeeping makes up a bigger part of the implementation.
\\\\ 
This initial version of quicksort works but it has one major drawback. The procedure is designed to only sort a single global variable. However
it is not possible to sort arbitrary arrays e.g. arrays that are passed as input argument to the procedure. Initially we overcame this shortcomming
by introducing more global array variables and having multiple copies of the same qucksort procedure, each working on a different global variable.
To avoid having to do this for each new array that has to be sorted, we came up with the following solution. First the \texttt{qs} procedure was changed
to sort a global array variable called \texttt{a\_qs}. Then to be able to sort arbitrary arrays, we 
introduced the following new procedure:

\begin{verbatim}
procedure quickSort(arr : [int]int, lo : int, hi : int) 
  returns (arr_sorted : [int]int, perm: [int]int) 
  modifies a_qs;
{
  // write input array into a_qs
  a_qs := arr;
  // let quicksort implementation sort a_qs
  call perm := qs(lo,hi);
  // write a_qs into output argument
  arr_sorted := a_qs;
}
\end{verbatim}

This procedure uses \texttt{a\_qs} as temporary variable for sorting arbitrary arrays. \texttt{quickSort} takes an array 
as input argument, copies it into \texttt{a\_qs}, calls \texttt{qs} to sort it and then writes the now sorted array into
the return value. Using this construction we are now able to sorte arbitrary arrays by having just one global variable.


\subsection{Bucketsort Implementation}
For our bucketsort implementation we decided to copy the procedure signature from quicksort. 

\begin{verbatim}
procedure bucketSort(lo : int, hi : int) returns (perm: [int]int)
  ...
}
\end{verbatim}

We know that \texttt{bucketSort} is only called with array elements that range from $-3*N$ to $3*N$ therefore
the algorithm divides the array to be sorted into three buckets with elements in the range $[-3*N)$, $[-N,N)$ and
$[N,3*N)$ respectively. The three buckets are implemented as three arrays. \texttt{bucketSort} iterates over the original
array and copies each element to it's corresponding bucket array. Then each bucket is sorted using our existing qucksort implementation.
After that, the now sorted bucket arrays are written back to the original array yielding a sorted version of the original array. 
\\\\
Unfortunately we didn't have the time to implement the construction of a permutation that represents how the elements of the original array were 
permuted by \texttt{bucketSort}. This task proved to be more challenging than in the case of quicksort. Because the original array is divided into 
three buckets in a single while loop and we can't know how many elements will end up in each bucket it's hard to construct a permutation that reflects
how the array elements are divided into the three buckets.

\subsection{Specification}

Both sort functions share the following specification:

\begin{verbatim}
procedure sort(lo : int, hi : int) returns (perm: [int]int)
  modifies a;
  requires lo <= hi;
  // perm is a permutation
  ensures (forall i: int :: lo <= i && i <= hi ==> lo <= perm[i] && perm[i] <= hi);
  ensures (forall k, l: int :: lo <= k && k < l && l <= hi ==> perm[k] != perm[l]);
  // the final array is that permutation of the input array
  ensures (forall i: int :: lo <= i && i <= hi ==> a[i] == old(a)[perm[i]]);
  // array is sorted
  ensures (forall k, l: int :: lo <= k && k <= l && l <= hi  ==> a[k] <= a[l]);
{ ... }
\end{verbatim}

Each sort procedure sorts some global array in the index range given by $lo$ and $hi$. 
There is only one precondition, namely that this range is not negative. The postconditions guarantee that the
array is sorted and that the final array is a permutation of the original. To express this property, the procedure
returns a concrete permutation that satisfies this property. 
We decided to use this specification, because it was already sucessfully used to 
specify and verify bubblesort in an example boogie code provided by microsoft research 
(TODO cite http://rise4fun.com/Boogie/Bubble). A consequence of this choice of specification is that the sort procedure has
to construct a permutation. Alternatively we could have just stated the existance of such a permutation. This however would have
been harder to prove in our opinion. The advantage of our approach is that the implementation of the procedure servers as a constructive
proofe for the existance of such a permutaion. In a way this allows us to state the prove rather than having to depend on boogies ability to 
infer the existinance of a permutation somehow.
\\\\
How these permutations are constructed was already discussed in sections X and Y (TODO reference sectinos).

\subsection{Verification}
TODO: report "significant problems" (e.g. which procedures could be verified, and which could not\\
TODO: describe changes to implementation/specification made to simplify proofs\\
TODO: describe which parts of the specification you could not verify and why\\
TODO: explain how you achieved modular verification

\subsection{Boogie vs. Autoproof}
