\section{Autoproof}\label{eiffel_discussion}
We are given a class
\begin{eBox}
class
	SV_AUTOPROOF

feature 
	lst: SIMPLE_ARRAY [INTEGER]
\end{eBox}

And we will specify its features below in a way such that Autoproof can verify them.\\
We refer to Appendix(\ref{eiffel_code}) for the complete code that we modify.

\subsection{wipe}
Wipe(\ref{eiffel_code_wipe}) takes an array of integers and resets all its item to 0 
We add a loop invariant
\begin{eSimple}
across 1 |..| (k-1) as i all
	x.sequence [i.item] = 0 end
\end{eSimple}

This is sufficient, because the first postcondition,
\begin{eSimple}
x.count = old x.count
\end{eSimple}
is already maintained by another invariant.   

\subsection{mod\_three}
Procedure \emph{mod\_three} (\ref{eiffel_code_mod_three}) takes two integer arrays \emph{a,b} of equal length, uses of \emph{wipe} on both and returns \emph{b} with its every third element set to one.

First, we maintain that the amount of integers in \emph{b} does not change:
\begin{eSimple}
b.count = b.count.old_
\end{eSimple}
This invariant is necessary. Without it, the assignment
\begin{eSimple}
b[k] := a[k] + 1
\end{eSimple}
may be out of bounds from Autoproof's point of view, because we iterate over the length of \emph{a}, which is specified to be constant and only initially equal to the length of \emph{b}.\\

Then, we need to postulate that each iteration over the loop can by itself change \emph{b}:
\begin{eSimple}
modify(b)
\end{eSimple}
Omitting this invariant will lead to Autoproof's insisting that \emph{b} has never changed and that any further invariant claiming otherwise couldn't possibly be maintained.\\

Having specified that, we can now add an invariant
\begin{eSimple}
across 1 |..| (k-1) as i all
	(i.item \\ 3 = 0) implies
		b.sequence [i.item] = 1 end
\end{eSimple}
which will claim that every third item we already iterated over in a is one in b. The assignment may read
\begin{eSimple}
b [k] := a[k] + 1
\end{eSimple}
but the postcondition from \emph{wipe(a)} allows Autoproof to deduce the element's being set to one without any further specification of ours.\\

\subsection{swapper}
\emph{Swapper}(\ref{eiffel_code_swapper}) relies on \emph{swap}(\ref{eiffel_code_swap}) to reverse \emph{lst} (which is global).\\

The loop here goes
\begin{eSimple}
from
	x := 1
	y := lst.count
until
	y <= x
\end{eSimple}

and after each iteration, x is incremented by one and y is decremented by one. This allows us to use \emph{y > 0} as a way of specifying an invariant that trivially holds before \emph{y := lst.count} has been executed and specifies some useful property afterwards - in our case, we use it to specify that once initialised, both \emph{x} and \emph{y} are within the bounds of \emph{lst}, and therefore satisfy the precondition of \emph{swap}.
\begin{eSimple}
y > 0 implies (1 <= x and x <= lst.count and 1 <= y and y <= lst.count)
\end{eSimple}

For Autoproof to be able to proof that the swapped list is a permutation of the original list, we need to specify that all items not swapped remained the same. \emph{Swap} itself does provide such a postcondition, however, this is insufficient because the \emph{old lst} \emph{swap}'s postcondition is mentioning is in fact the \emph{lst} at the moment \emph{swapper} is calling \emph{swap}, which changes with each iteration. We must link these two \emph{"olds"} explicitly:
\begin{eSimple}
across x |..| y as i all lst.sequence[i.item] =
	lst.sequence.old_[i.item] end
\end{eSimple}

Swapper's postcondition states
\begin{eSimple}
across 1 |..| lst.count as i all lst.sequence [i.item] = (old lst.sequence) [lst.count - i.item + 1] end
\end{eSimple}
This directly motivates the addition of the following two loop invariants:
\begin{eSimple}
x > 1 implies across 1 |..| (x-1) as i all lst.sequence[i.item] = lst.sequence.old_[lst.count-i.item + 1] end

x > 1 implies across 1 |..| (x-1) as i all lst.sequence[lst.count-i.item+1] = lst.sequence.old_[i.item] end
\end{eSimple}
Where we are again using \emph{x > 1} as a way of saying "x and y have both been initialised". We need to split the postcondition into two parts because the items in-between \emph{x} and \emph{y} have not been swapped yet.\\

However, we are using \emph{lst.count-i.item+1} so that it matches the postcondition and Autoproof cannot proof that just yet because it's never actually using this expression in the loop. What we thus require is another invariant that specifies the index the loop is using to correlate to the arithmetic in the postcondition and the invariants:
\begin{eSimple}
y = lst.count - x + 1
\end{eSimple}

\subsection{search}
\emph{Search} (\ref{eiffel_code_search}) is traversing \emph{lst} backwards and returns \emph{True} iff it contains search key \emph{v}. We need to specify both, postconditions and invariants, so we start with the postconditions to help us find the invariants we need.\\

First, we specify that \emph{lst}, who is required to be wrapped, will remain so.
\begin{eSimple}
lst.is_wrapped
\end{eSimple}
Since \emph{search} does not contain a \emph{modify}-clause for \emph{lst}, we do not need to explicitly specify that it does not change \emph{lst}.\\

We then specify the actual return value:
\begin{eSimple}
Result implies across 1 |..| lst.count as i some lst.sequence[i.item] = v end
(not Result) implies across 1 |..| lst.count as i all lst.sequence[i.item] /= v end
\end{eSimple}
We could have replaced these two implications by an equality
\begin{eSimple}
Result = across 1 |..| lst.count as i some lst.sequence[i.item] = v end
\end{eSimple}
but not doing so allowed us to check each direction individually, which helped in finding the required invariants.\\

The loop here goes
\begin{eSimple}
from
	k := lst.count
	Result := False
until
	Result or k < 1
\end{eSimple}
We first specify \emph{k} to remain within the allowed range of indices for \emph{lst}, using \emph{k > 0} to ignore its value prior to entering the loop and after exiting the loop when \emph{lst} does not contain \emph{v}:
\begin{eSimple}
k > 0 implies (1 <= k and k <= lst.count)
\end{eSimple}

Because \emph{Result} is only ever set when the current iteration finds \emph{v} and \emph{k} is only decremented when we do not, specifying the case where we do find \emph{v} becomes easy:
\begin{eSimple}
Result implies (lst.sequence[k] = v)
\end{eSimple}
To specify the case where we do not find \emph{v}, we take into account that \emph{k} is being decremented, starting from the last valid index down to zero, which means at any time during the iteration, all elements with a valid index greater than \emph{k} have been checked not to be \emph{v}.
\begin{eSimple}
(not Result) implies (across (k+1) |..| lst.count as i all lst.sequence[i.item] /= v end)
\end{eSimple}

\subsection{prod\_sum}
Prod\_sum (\ref{eiffel_code_prod_sum}) was a less complicated matter. In fact, copying the postcondition proved sufficient:
\begin{eSimple}
zz * y + xx = xx.old_
\end{eSimple}

\subsection{paly}
Paly (\ref{eiffel_code_paly}) takes an integer array and returns \emph{True} iff its elements form a palindrome (i.e. represent an integer string whose every prefix is a reversed suffix). We need to specify both, postconditions and invariants, and we start again with the postconditions to help us find the invariants:\\
\begin{eSimple}
Result implies across 1 |..| a.count as i all a.sequence[i.item] = a.sequence[a.count-i.item + 1] end
(not Result) implies across 1 |..| a.count as i some a.sequence[i.item] /= a.sequence[a.count-i.item+1] end
\end{eSimple}
As with \emph{search}, we could have expressed these two implications as a single equality but choose not to for easier reasoning about it.\\

The loop here goes
\begin{eSimple}
from
	x := 1
	y := a.count
	Result := True
until
	x >= y or not Result
\end{eSimple}
We first specify \emph{x} and \emph{y} to remain within the bounds of the array, using \emph{y > 0} to avoid violation on entry.
\begin{eSimple}
y > 0 implies (1 <= x and x <= a.count and 1 <= y and y <= a.count)
\end{eSimple}

We then define a different interpretation for \emph{y} to allow its translation into the arithmetic expected by the postconditions.
\begin{eSimple}
y = a.count - x + 1
\end{eSimple}

\emph{Result} is initialised to \emph{True} and set to false as soon as we encounter a prefix that is not a reversed suffix. We exploit the fact that \emph{x} is incremented with every iteration to express that \emph{Result = True} implies "after at least one iteration" that the prefix iterated over so far is a reversed suffix:
\begin{eSimple}
(x > 1 and Result) implies across 1 |..| (x-1) as i all a.sequence[i.item] = a.sequence[a.count-i.item + 1] end
\end{eSimple}
Because \emph{Result} is iterated to true, we need no such trick for the other direction.
\begin{eSimple}
(not Result) implies across 1 |..| a.count as i some a.sequence[i.item] /= a.sequence[a.count-i.item + 1] end
\end{eSimple}