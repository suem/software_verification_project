\section{Introduction}
Producing provably correct programs is becoming more and more important as society is being automatized and software is written that has control over safety-critical components, e.g. a car's brakes.\\

Proving a program correct is a tedious task, which leads to both, a high demand for tools assisting such proofs and an increased interest in research driving the development of these tools. The ETH's master course in \emph{Software Verification} is - at the time of writing - encouraging students taking the course to complete a couple tasks requiring them to experiment with two notable verification tools: AutoProof\cite{autoproof} for Eiffel and Boogie\cite{boogie} for the verification language with the same name.\\

In a first part of the project, the students are asked, this year, to complete an Eiffel program such that AutoProof can verify it. The program can be found in the appendix (\ref{eiffel_code}). We discuss our solution in \autoref{eiffel_discussion} and provide full code for our solution in \autoref{eiffel_code_solution}.\\

A second part of the project consists of modeling a sorting algorithm that alternates between quick- and bucketsort, depending on the elements in the array passed to it, in Boogie. The appendix holds a more detailed description of the algorithm in form of a Boogie template (\ref{boogie_code}). We discuss our approach in \autoref{boogie_discussion}, along with some issues and interesting behaviours we came across and provide full code of our solution in \autoref{boogie_code_solution}.
